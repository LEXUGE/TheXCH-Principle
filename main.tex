\documentclass[1pt]{article}
\usepackage{geometry}
\geometry{left=4.0cm,right=4.0cm,top=3.0cm,bottom=3.0cm}
\usepackage{fontspec}
\usepackage{amsmath}
\usepackage{amssymb}
\usepackage{listings}
\usepackage{xcolor}
\usepackage[colorlinks,linkcolor=black,urlcolor=blue]{hyperref}
\lstset{
    columns=fixed,
    numbers=left,                                        % 在左侧显示行号
    frame=none,                                          % 不显示背景边框
    backgroundcolor=\color[RGB]{255,255,255},            % 设定背景颜色
    keywordstyle=\color[RGB]{40,40,255},                 % 设定关键字颜色
    numberstyle=\footnotesize\color{darkgray},           % 设定行号格式
    stringstyle=\rmfamily\slshape\color[RGB]{128,0,0},   % 设置字符串格式
    showstringspaces=false,                              % 不显示字符串中的空格
    language=C,                                        % 设置语言
}
\usepackage[BoldFont, SlantFont, CJKchecksingle]{xeCJK}
\setmainfont{DejaVu Sans}
\setCJKmainfont[BoldFont={WenQuanYi Micro Hei/Bold}]{WenQuanYi Micro Hei}

\title{XCH配平原理与算法}
\author{Ying Kanyang (LEXUGE) <LEXUGEyky@outlook.com>}
\date{\today}

\begin{document}
  \maketitle
  \newpage
  \tableofcontents
  \newpage
  \section{XCH简介}
    XCH是自动解析并配平化学方程式的套件,由\href{https://crates.io/crates/lib_xch}{lib\_xch}与\href{https://crates.io/crates/xch-ceb}{xch-ceb}组成。使用Rust语言编写。其中所有的实现算法都在lib\_xch中实现,并暴露API,xch-ceb是lib\_xch的前端。目标是“轻巧,快速,安全”。\\
    本文将介绍XCH的原理与算法,并相应地给出其实现。并给出目前存在的缺陷,以及未来的计划。
  \subsection{lib\_xch}
    lib\_xch主要组成部分是Delta-3解析模块(mod)与AlphaForce配平模块。Delta-3解析模块使用的Regex分词提取Token的方法,并将各元素在各分子式的个数转换为一张表(table)。其中,使用了从内到外拆分的方式来实现多层括号的嵌套。作为解析模块的一部分,也作了相应的语法检查。\\
    AlphaForce配平模块先是将Delta-3解析模块的结果转换为方程组(Equation Set),并使用矩阵的方式表示出来,使用了高斯-约当消元算法(The Gaussian-Jordan Algorithm)解出各项系数,从而得出结果。\\
    当然,由于化学方程式并不是都能转换为拥有唯一解的方程组,因此需要做一部分的优化以及处理。并且,由于算法本身的特殊性,计算过程中可能产生分数,需要设计专门的数据结构来处理,以达到“零精度损失”的目标。
  \subsection{xch-ceb}
    xch-ceb是XCH的前端,负责调用lib\_xch提供的API,并处理用户的输入,最后返回lib\_xch的结果。
  \section{原理及算法}
  \subsection{原理}
      先来看一个未配平的化学方程式:
    \begin{equation}
      As_{2}O_{3}+Zn+HCl=AsH_{3}+ZnCl_{2}+H_{2}O
    \end{equation}
    设每个化学式(Chemical Formula)前的系数为未知数,可得:
    \begin{equation}
      x_{1}As_{2}O_{3}+x_{2}Zn+x_{3}HCl=x_{4}AsH_{3}+x_{5}ZnCl_{2}+x_{6}H_{2}O
    \end{equation}
    根据原子个数守恒(the law of conservation of atoms),可将上述方程转换为关于$x$的多元一次方程组:\\
    \begin{equation}
      \left\{
        \begin{aligned}
          2x_{1}+0x_{2}+0x_{3}&=1x_{4}+0x_{5}+0x_{6}\cdots{}As\\
          3x_{1}+0x_{2}+0x_{3}&=0x_{4}+0x_{5}+1x_{6}\cdots{}O\\
          0x_{1}+1x_{2}+0x_{3}&=0x_{4}+1x_{5}+0x_{6}\cdots{}Zn\\
          0x_{1}+0x_{2}+1x_{3}&=3x_{4}+0x_{5}+2x_{6}\cdots{}H\\
          0x_{1}+0x_{2}+1x_{3}&=0x_{4}+2x_{5}+0x_{6}\cdots{}Cl\\
        \end{aligned}
      \right.
    \end{equation}
    整理得:\\
    \begin{equation}
      \left\{
        \begin{aligned}
          2x_{1}-x_{4}&=0\cdots{}As\\
          3x_{1}-x_{6}&=0\cdots{}O\\
          x_{2}-x_{5}&=0\cdots{}Zn\\
          x_{3}-3x_{4}-2x_{6}&=0\cdots{}H\\
          x_{3}-2x_{5}&=0\cdots{}Cl\\
        \end{aligned}
      \right.
    \end{equation}
    显然,方程组存在多解。将其转换为增广矩阵的形式:\\
    \begin{equation}
      A=\left [
        \begin{array}{cccccc|c}
          2 & 0 & 0 & -1 & 0 & 0 & 0 \\
          3 & 0 & 0 & 0 & 0 & -1 & 0 \\
          0 & 1 & 0 & 0 & -1 & 0 & 0 \\
          0 & 0 & 1 & -3 & 0 & -2 & 0 \\
          0 & 0 & 1 & 0 & -2 & 0 & 0 \\
        \end{array}
      \right]
    \end{equation}
    对其进行高斯-约当消元,得到:\\
    \begin{equation}
      A=\left [
        \begin{array}{cccccc|c}
          1 & 0 & 0 & 0 & 0 & -\frac{1}{3} & 0 \\
          0 & 1 & 0 & 0 & -1 & 0 & 0 \\
          0 & 0 & 1 & -3 & 0 & -2 & 0 \\
          0 & 0 & 0 & 1 & 0 & -\frac{2}{3} & 0 \\
          0 & 0 & 0 & 0 & 1 & -2 & 0 \\
        \end{array}
      \right]
    \end{equation}
    自由元(Free Variable)$x_{6}$,设$n$个自由元中任意一个为1,其余为0,并求出所有未知数。通过$n$组解的线性表示,就可以得出所有的可能解,也就是所有可能的系数。\\
    这里,设所有的自由元均为$1$\footnote{实际情况中,因为系数不为0,所以设自由元为1最可能得到一般性的化学方程式},得到解为:\\
    \begin{equation}
      \left\{
        \begin{aligned}
          x_{1}&=\frac{1}{3}\\
          x_{2}&=2\\
          x_{3}&=4\\
          x_{4}&=\frac{2}{3}\\
          x_{5}&=2\\
          x_{6}&=1\\
        \end{aligned}
      \right.
    \end{equation}
    因为系数为整数,对其化整:\\
    \begin{equation}
      \left\{
        \begin{aligned}
          x_{1}&=1\\
          x_{2}&=6\\
          x_{3}&=12\\
          x_{4}&=2\\
          x_{5}&=6\\
          x_{6}&=3\\
        \end{aligned}
      \right.
    \end{equation}
    所以,解为:
    \begin{equation}
      As_{2}O_{3}+6Zn+12HCl=2AsH_{3}+6ZnCl_{2}+3H_{2}O
    \end{equation}
  \subsection{解析模块设计}
      但是,为了让用户输入的化学方程式转换为增广矩阵的形式。需要设计解析器(Parser)来对方程式进行解析。\\
      设计的思路是:\\
      1. 方程式按照等号拆成两边。\\
      2. 按照$+$拆为化学式。\\
      3. 使用正则表达式对最内层的括号单位(稍后介绍)进行拆解\\
      4. 重复3直到无法匹配到括号单位\\
      5. 对单个处理过的化学式记录到表中\\
      6. 记录完所有的化学式\\
    这样,就完成了解析的工作。\\
    \subsubsection{正则表达式设计}
        括号单位,我们定义为形如这样的化学式:($t$为$1$时可省略)\\
      \begin{equation}
        (A_{1}c_{1}A_{2}c_{2}\cdots{}A_{n}c_{n})_{t}
      \end{equation}
      如$(OH)_{2}$就属于括号单位。括号单位的正则表达式是:\\
      \begin{lstlisting}
        \((([A-Z][a-z]*(\d+)*)+)\)(\d+)*
      \end{lstlisting}
      此外,为了在记录时便于获取各元素的系数,还可以使用如下正则表达式匹配元素与系数:\\
      \begin{lstlisting}
        ([A-Z][a-z]*)(\d+)*
      \end{lstlisting}
  \subsection{配平模块设计}
      配平模块主要是对原理的计算机实现,主要难点在于高斯-约当算法与如何实现“零精度误差”的分数数据结构。
    \subsubsection{高斯-约当消元的实现细节}
        直接实现的高斯-约当消元是极低效率且难以工作的。在实际实现中,需要注意以下两个特性细节的添加:\\
        1. 对于当前行(第$p$行),与$p+1$到$n$行中主元\footnote{当前行第一个非零元素}(pivot)最左的行交换。\\
        2. $p+1$到$n$行,交换后的当前行主元所在列中,选择绝对值最大的行交换。
    \subsubsection{分数数据结构细节}
        需要实现的有基本的加减乘除运算以及化简,返回实数,比较大小,绝对值等。
  \section{代码实现}
    下面所有代码已被折行,标准格式请在\href{https://github.com/LEXUGE/lib-xch-ceb}{GitHub}上查看。
    \subsection{Delta-3代码核心解析}
    将表(table)作为对象来实现:\\
      \begin{lstlisting}
      pub struct TableDesc {
          elements_table: HashMap<String, usize>, // 元素哈希检索表
          list: Vec<Vec<i32>>, // 存储表
          formula_sum: i32, // 化学式个数
      }
      \end{lstlisting}
      impl中的核心方法(method)是store\_in\_table:
      \begin{lstlisting}
      pub fn store_in_table(
        &mut self,
        formula: &str,
        location: usize
      ) -> Result<bool, ErrorCases> {
          for t in get_token(formula)? {
              if !self.elements_table
                .contains_key(&t.token_name) {
                  // 检查元素是否已经出现
                  let len = self.elements_table.len();
                  self.elements_table.insert(
                      t.token_name.clone(),
                      len + 1,
                  );
                  self.update_list_vec();
              }

              {
                  // 向表中写入数据
                  let tmp = match self.elements_table.get(
                      &t.token_name
                  ) {
                      Some(s) => *s,
                      None => return Err(NotFound),
                  };
                  self.list[tmp][location]
                    = match self.list[tmp][location]
                      .checked_add(t.times) {
                      Some(s) => s,
                      None => return Err(I32Overflow),
                  }
              }
          }
          Ok(true)
      }
      \end{lstlisting}
      此外就是对括号进行拆解的功能(function):
      \begin{lstlisting}
      fn parser_formula(
        // 解析化学式
        formula: &str,
        table: &mut TableDesc,
        location: usize,
      ) -> Result<bool, ErrorCases> {
        let formula_backup = formula;
        let mut formula = format!("({})", formula_backup);
        // 对于方程式左右加上"(",")",使其满足括号单位的定义

        formula_spliter(&formula)?;
        while formula_spliter(&formula).is_ok() {
            for p in formula_spliter(&formula)? {
                // 每次拆分最内层的括号单位并替换
                formula = replace_phrase(
                  &formula,
                  &p.all,
                  &(mul_phrase(&p)?)
                );
            }
        }
        table.store_in_table(&formula, location)?;
        Ok(true)
      }
      \end{lstlisting}
    \subsection{AlphaForce代码核心解析}
      分数数据结构的实现就不在此赘述,只需要模拟即可,注意需要对每一步运算做溢出检测。
      \subsubsection{高斯-约当消元算法实现}
        这里介绍两个核心的结构设计:
        \begin{lstlisting}
          fn get_leftmost_row(&self, row: usize) -> Option<usize> {
          let mut fake_zero = false; // “零锁”设计
          let mut leftmost = row;
          let mut min_left: usize = match self.get_pivot(row) {
              Some(s) => s,
              None => {
                  fake_zero = true;
                  0
              }
          };
          for i in row + 1..self.n {
              let current_pivot = match self.get_pivot(i) {
                  Some(s) => s,
                  None => continue, // 有全0行就跳过
              };
              if (current_pivot < min_left) | (fake_zero) {
                  // 只要fake_zero为true就会替换,实现了零锁行最大
                  leftmost = i;
                  min_left = current_pivot;
                  fake_zero = false;
              }
          }
          if fake_zero {
              None // 如果零锁仍存在说明全部为全0行
          } else {
              Some(leftmost)
          }
        }
        \end{lstlisting}
        零锁设计保证了代码在$p+1$到$n$行全为0时有返回None而当任何一行不为全0时替换。\\
        其余的部分就是对高斯-约当算法的实现,只需要学习线性代数即可。这里是实现:\\
        \begin{lstlisting}
        pub fn solve(&mut self) -> Result<
          ResultHandler<Vec<Frac>>,
          ErrorCases
        > {
          // The Gaussian-Jordan Algorithm
          for i in 0..self.n {
              let leftmosti = match self.get_leftmost_row(i) {
                  Some(s) => s,
                  None => continue,
              };
              self.matrix_a.swap(i, leftmosti);
              self.matrix_b.swap(i, leftmosti);
              let j = match self.get_pivot(i) {
                  // 如果“最左”行依旧为0,跳过
                  Some(s) => s,
                  None => continue,
              };
              let maxi = self.get_max_abs_row(i, j)?;
              if self.matrix_a[maxi][j].numerator != 0 {
                  self.matrix_a.swap(i, maxi);
                  self.matrix_b.swap(i, maxi); // 交换绝对值大的行
                  {
                      let tmp = self.matrix_a[i][j];
                      self.divide_row(i, tmp)?;
                  }
                  for u in i + 1..self.n {
                      let v = self.mul_row(i, self.matrix_a[u][j])?;
                      for (k, item) in v.iter().enumerate()
                        .take(self.m) {
                          self.matrix_a[u][k]
                            = self.matrix_a[u][k].sub(*item)?;
                      }
                      self.matrix_b[u]
                        = self.matrix_b[u].sub(v[self.m])?;
                  }
              }
          } // 行梯阵式(REF)

          for i in (0..self.n).rev() {
              let j = match self.get_pivot(i) {
                  Some(s) => s,
                  None => continue,
              };
              for u in (0..i).rev() {
                  // j above i
                  let v = self
                    .mul_row(i, self.matrix_a[u][j])?;
                  for (k, item) in v.iter().enumerate()
                    .take(self.m) {
                      self.matrix_a[u][k] = self.matrix_a[u][k]
                        .sub(*item)?;
                  }
                  self.matrix_b[u] = self.matrix_b[u].sub(v[self.m])?;
              }
          } // 简化行梯阵式(RREF)
          let mut ans: Vec<Frac> = vec![Frac::new(0, 1); self.m];
          let pivots = self.check()?;
          let mut free_variable = false;
          for i in (0..self.m).rev() {
              if pivots.contains_key(&i) {
                  let mut sum = Frac::new(0, 1);
                  for (k, item) in ans.iter().enumerate()
                    .take(self.m).skip(i + 1) {
                      sum = sum.add(
                          self.matrix_a[pivots[&i]][k].mul(*item)?
                        )?;
                  }
                  ans[i] = self.matrix_b[pivots[&i]]
                      .sub(sum)?
                      .div(self.matrix_a[pivots[&i]][i])?;
              } else {
                  free_variable = true;
                  ans[i] = Frac::new(1, 1); // 设所有的自由元为1
              }
          }
          Ok(ResultHandler {
              warn_message: if free_variable {
                  FreeVariablesDetected
              } else {
                  NoWarn
              },
              result: ans,
          })
        }
        \end{lstlisting}
  \newpage
  \section{总结}
    lib\_xch实现了化学方程式的配平基本需求,但是还不支持离子方程式等。离子方程式的配平原理与此也相同。此方法适用于大部分的方程式,但是还有部分缺陷,如无法配平此方程:\\
  \begin{equation}
    KClO_{3}+HCl=KCl+ClO_{2}+Cl_{2}+H_{2}O
  \end{equation}
  使用ILP模型\footnote{ILP模型属于NP完全问题,时间复杂度不可预估,所以没有采用}可以配平:\\
  \begin{equation}
    2KClO_{3}+4HCl=2KCl+2ClO_{2}+Cl_{2}+2H_{2}O
  \end{equation}
  此方法还可以用于猜测新的化学方程式(使用已知的线性无关解来线性表示),等等。\\
    库(Library)运用了一些Rust的特性,在我编写的过程中也真切地体会到Rust的用户友好(user-friendly),不但无需担心一些内存的分配与回收,而且还能在没有运行时(runtime)的情况下实现完整的异常处理(使用Result的可恢复错误机制)。语言本身还结合了面向对象编程(Object-Obriented Programming)与函数式编程的特点(如Monad,Lambda等),是一门真正现代设计的语言。\\
    与此同时Rust还天生提供了媲美C的运行速度以及优化,让库的运行效率提高不少。\\
    但是,Rust也有其缺点,如:编译效率低,生态不完全(尽管现在已经发展了不少),许多特性依赖编译等。不过,这对我而言算不了什么,因此,我选择了Rust。的确,这也是一个正确的选择。\\
  \\
  \\
  任何BUG,意见或建议欢迎发邮件至\href{mailto:LEXUGEyky@outlook.com}{LEXUGEyky@outlook.com}或在GitHub上开\href{https://github.com/LEXUGE/lib-xch-ceb/issues}{issue}。

  \newpage
  \begin{thebibliography}{1}
  \bibitem{1} Blakley, G. R. (1982). Chemical equation balancing. Journal of Chemical Education, 59.\\
  \bibitem{2} Sen, S. K., Agarwal, H., \& Sen, S. (2006). Chemical equation balancing: An integer programming approach. Elsevier Science Publishers B. V.\\
  \end{thebibliography}
\end{document}
